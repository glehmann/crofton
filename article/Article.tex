%
% Complete documentation on the extended LaTeX markup used for Insight
% documentation is available in ``Documenting Insight'', which is part
% of the standard documentation for Insight.  It may be found online
% at:
%
%     http://www.itk.org/

\documentclass{InsightArticle}


%%%%%%%%%%%%%%%%%%%%%%%%%%%%%%%%%%%%%%%%%%%%%%%%%%%%%%%%%%%%%%%%%%
%
%  hyperref should be the last package to be loaded.
%
%%%%%%%%%%%%%%%%%%%%%%%%%%%%%%%%%%%%%%%%%%%%%%%%%%%%%%%%%%%%%%%%%%
\usepackage[dvips,
bookmarks,
bookmarksopen,
backref,
colorlinks,linkcolor={blue},citecolor={blue},urlcolor={blue},
]{hyperref}
% to be able to use options in graphics
\usepackage{graphicx}
% for pseudo code
\usepackage{listings}
% subfigures
\usepackage{subfigure}


%  This is a template for Papers to the Insight Journal. 
%  It is comparable to a technical report format.

% The title should be descriptive enough for people to be able to find
% the relevant document. 
\title{N-Dimensional surface estimation using the Crofton formula and the run-length encoding}

% Increment the release number whenever significant changes are made.
% The author and/or editor can define 'significant' however they like.
% \release{0.00}

% At minimum, give your name and an email address.  You can include a
% snail-mail address if you like.
\author{Ga\"etan Lehmann{$^1$} {\small{and}} David Legland{$^2$}}
\authoraddress{{$^1$}INRA, UMR 1198; ENVA; CNRS, FRE 2857, Biologie du D\'eveloppement et Reproduction, Jouy en Josas, F-78350, France.\\
{$^2$}INRA, XXX, France.}

\begin{document}
\maketitle

\ifhtml
\chapter*{Front Matter\label{front}}
\fi


\begin{abstract}
\noindent
Unlike the measure of the area in 2D or of the volume in 3D, the perimeter and the surface are not easily measurable in a discretized image.
In this article we describe a method based on the Crofton formula to measure those two parameters in a discritized image. The accuracy of
the method is discussed and tested on several known objects. An algorithm based on the run-length encoding of binary objects is presented
and compared to a brute force approach.
An implementation is provided and integrated in the LabelObject/LabelMap framework contributed earlier by the authors.
\end{abstract}

\tableofcontents

\section{la methode}

% à faire : trouver un titre et remplir cette section
% quelques idées de blabla:
\subsection{Other known methods}
\subsection{The Crofton formula}
\subsection{Theoritical accuracy}

\subsection{Counting the intercepts}

The number of intercepts, required to use the method described earlier, can be efficiently computed by using the run-length encoding of
the binary objects.

% à faire

\section{Implementation}

The implementation of this algorithm has been done in the \verb$itk::ShapeLabelMapFilter$ class, which is already in charge of the computation of
several shape descriptors. The run-length encoding used in the \verb$itk::LabelObject$ representation is reused.
The implementation is N-dimensional and thus is usable for any image dimension. The most useful cases, 2D and 3D have been specialized
to provide a more accurate estimation by also using the diagonals -- 4 directions in 2D and 13 directions in 3D. In the 4D case and greater,
the diagonals are not used.

\subsection{N-Dimensional name}
The attribute is called {\em Perimeter} independently of the dimension of the image and is available in the \verb$itk::ShapeLabelObject$.

\subsection{Border management}

The borders of the image is considered to be in the background, so the contour of an object touching the border of the image is measured.
This was not the case in the previous (undocumented) implementation and has been proven to be misleading for many users.
It is possible to subtract the perimeter on the border to the full perimeter if the measure without the part touching the border is needed.

\subsection{Multithreading}

The architecture implemented in \verb$itk::ShapeLabelMapFilter$ use one thread per LabelObject. The perimeter estimation has been integrated in this
architecture, providing a multithreaded implementation as long as there are several LabelObjects in the input LabelMap.

\subsection{Accuracy}

% à faire
% à mettre l'impact du calcul surface - surface sur le bord

\subsection{Performance}

For comparison, we provide two implementations of the brute force algorithm -- one dedicated to a single binary object, and another for
several, labelized objects. The implementation for labelized objects is slower but closer of the capabilities of the run-length encoding based
implementation.

% à faire

\subsection{Usage}

The perimeter estimation is now turned on by default, but can still disabled with \verb$SetComputePerimeter(false)$ in
\verb$itk::ShapeLabelMapFilter$ if not needed. This should enhance the user experience, especially for the attributes non obviously derived
from the perimeter like the roundness.



\appendix



\bibliographystyle{plain}
\bibliography{InsightJournal}
\nocite{ITKSoftwareGuide}

\end{document}

